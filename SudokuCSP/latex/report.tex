% !TEX TS-program = pdflatex
% !TEX encoding = UTF-8 Unicode

% This is a simple template for a LaTeX document using the "article" class.
% See "book", "report", "letter" for other types of document.

\documentclass[11pt]{article} % use larger type; default would be 10pt

\usepackage[utf8]{inputenc} % set input encoding (not needed with XeLaTeX)

%%% Examples of Article customizations
% These packages are optional, depending whether you want the features they provide.
% See the LaTeX Companion or other references for full information.

%%% PAGE DIMENSIONS
\usepackage{geometry} % to change the page dimensions
\geometry{a4paper} % or letterpaper (US) or a5paper or....
% \geometry{margin=2in} % for example, change the margins to 2 inches all round
% \geometry{landscape} % set up the page for landscape
%   read geometry.pdf for detailed page layout information

\usepackage{graphicx} % support the \includegraphics command and options

% \usepackage[parfill]{parskip} % Activate to begin paragraphs with an empty line rather than an indent

%%% PACKAGES
\usepackage[font=small,format=plain,labelfont=bf,up,textfont=it,up]{caption} % for nice captions
\usepackage{booktabs} % for much better looking tables
\usepackage{array} % for better arrays (eg matrices) in maths
\usepackage{paralist} % very flexible & customisable lists (eg. enumerate/itemize, etc.)
\usepackage{verbatim} % adds environment for commenting out blocks of text & for better verbatim
\usepackage{subfig} % make it possible to include more than one captioned figure/table in a single float
% These packages are all incorporated in the memoir class to one degree or another...
\usepackage[table]{xcolor} % colors for the tables


%%% HEADERS & FOOTERS
\usepackage{fancyhdr} % This should be set AFTER setting up the page geometry
\pagestyle{fancy} % options: empty , plain , fancy
\renewcommand{\headrulewidth}{0pt} % customise the layout...
\lhead{}\chead{}\rhead{}
\lfoot{}\cfoot{\thepage}\rfoot{}

%%% SECTION TITLE APPEARANCE
\usepackage{sectsty}
\allsectionsfont{\sffamily\mdseries\upshape} % (See the fntguide.pdf for font help)
% (This matches ConTeXt defaults)

%%% ToC (table of contents) APPEARANCE
\usepackage[nottoc,notlof,notlot]{tocbibind} % Put the bibliography in the ToC
\usepackage[titles,subfigure]{tocloft} % Alter the style of the Table of Contents
\renewcommand{\cftsecfont}{\rmfamily\mdseries\upshape}
\renewcommand{\cftsecpagefont}{\rmfamily\mdseries\upshape} % No bold!

%%% BIBLIOGRAPHY
\usepackage{natbib}
\usepackage{url}


%%% END Article customizations

%%% The "real" document content comes below...

\title{Automated Reasoning in AI\\
Assignment 3: Description Logic}
\author{Armon Toubman \and Torec Luik}
%\date{} % Activate to display a given date or no date (if empty),
         % otherwise the current date is printed

\begin{document}
\maketitle

\section{Tableau algorithm}
\label{sec:tableau}

We use the following mapping from Description Logic to LoTREC. It is similar to the example given in the assignment except for some of the LoTREC displays.

\begin{table}[h]
\begin{center}
\begin{tabular}{l c c c c c c}
$\mathcal{ALC}$ syntax & $\neg C$ & $C \sqcap D$ & $C \sqcup D$ & $\exists r.C$ & $\forall r.C$ & $\top \equiv D$\\
LoTREC connective & \texttt{not} $C$ & \texttt{and} $C D$ & \texttt{or} $C D$ & \texttt{some} $r$ $C$ & \texttt{only} $r$ $C$ & \texttt{tbox} $D$\\
LoTREC display & $\sim$ $C$ & $C$ $^\wedge$ $D$ & $C$ v $D$ & $r$ some $C$ & $r$ only $C$ & $\top$ = $D$\\
\end{tabular}
\end{center}
\end{table}

To indicate $\top$ and $\bot$ we use two connectors \texttt{true} and \texttt{false}, both with arity 1. The actual variable used with these connectors is disregarded. The representation of lists of formulas and of problem instances are again much like the example:

\begin{table}[h]
\begin{center}
\begin{tabular}{l c c}
LoTREC connective & \texttt{add} $\phi$ $\psi$ & \texttt{input} $\mathcal{T}$ $C$\\
LoTREC display & $\phi$ \& $\psi$ & INPUT: TBox = $\mathcal{T}$; ABox = $C$\\
\end{tabular}
\end{center}
\end{table}

Because the only differences in our encoding and the example encoding in the assignment are in the LoTREC display, the LoTREC input of formulas is done in the same way as the example formula used in Figure 1 and 2 of the assignment:

\begin{table}[h]
\begin{center}
\begin{tabular}{c}
(\{$\top \equiv \exists s.C$, $\top \equiv \forall r.(\neg C \sqcup D)$\}, $\exists r.(C \sqcap D)$)\\
\texttt{input add tbox some S C tbox only R or not C D some R and C D}\\
\end{tabular}
\end{center}
\end{table}

All connectors, rules, strategies and example formulas can of course be found in our LoTREC file.

\section{Reasoning problems}
\label{sec:reasoning}

Solve the following decision problems supporting yourself with your implemen-
tation of tableau algorithm.

\subsection{}

\begin{center}
Is $\exists r.D$ satisfiable with regard to $\mathcal{T}$ = \{ $\top \equiv \exists s.C$, $\top \equiv \forall r.(\bot \sqcup E)$ \}?
\end{center}

\begin{itemize}

\item 1. State the problem.
\item 2. Reduce the problem to the corresponding concept satisability problem
(as shown during the lecture).
\item 3. Apply the necessary syntactic transformations to the TBox and the con-
cept so that the conditions (y), listed in the previous task, are satised.
\item 4. Translate the result into the input formula for your tableau implementa-
tion. Please, type and save this formula also in your implementation le
*.xml [LoTREC tab: Predefined Formulas]
\item 5. Use your implementation of the tableau to compute a tableau tree for this
formula.
\item 6. State the result of the computation (is the tableau closed or open?) If the
tableau is open include a picture of one of its open branches in the report
[LoTREC menu: Premodels/Export Premodel...].
4
\item 7. Based on the result of the computation provide the answers to the fol-
lowing problems: 1) the original decision problem; 2) the corresponding
concept satisability problem that you solved with the tableau algorithm.

\end{itemize}

\subsection{}

\begin{center}
Is $D \sqcap E$ subsumed by $\exists r.B$ in $\mathcal{T}$ = \{ $C \sqsubseteq \neg A$, $D \sqsubseteq \forall r.(A \sqcup B)$, $E \sqsubseteq \exists r.C$\}?
\end{center}

\begin{itemize}

\item 1. State the problem.
\item 2. Reduce the problem to the corresponding concept satisability problem
(as shown during the lecture).
\item 3. Apply the necessary syntactic transformations to the TBox and the con-
cept so that the conditions (y), listed in the previous task, are satised.
\item 4. Translate the result into the input formula for your tableau implementa-
tion. Please, type and save this formula also in your implementation le
*.xml [LoTREC tab: Predefined Formulas]
\item 5. Use your implementation of the tableau to compute a tableau tree for this
formula.
\item 6. State the result of the computation (is the tableau closed or open?) If the
tableau is open include a picture of one of its open branches in the report
[LoTREC menu: Premodels/Export Premodel...].
4
\item 7. Based on the result of the computation provide the answers to the fol-
lowing problems: 1) the original decision problem; 2) the corresponding
concept satisability problem that you solved with the tableau algorithm.

\end{itemize}

\subsection{}

\begin{center}
Is the ABox \{$C(a)$\} consistent with regard to $\mathcal{T}$ = \{$\top \equiv \forall r.B \sqcap \forall s.C$, $\top \equiv \neg \forall r.(\neg C \sqcap B)$, $\top \equiv \exists s.\top$\}?
\end{center}

\begin{itemize}

\item 1. State the problem.
\item 2. Reduce the problem to the corresponding concept satisability problem
(as shown during the lecture).
\item 3. Apply the necessary syntactic transformations to the TBox and the con-
cept so that the conditions (y), listed in the previous task, are satised.
\item 4. Translate the result into the input formula for your tableau implementa-
tion. Please, type and save this formula also in your implementation le
*.xml [LoTREC tab: Predefined Formulas]
\item 5. Use your implementation of the tableau to compute a tableau tree for this
formula.
\item 6. State the result of the computation (is the tableau closed or open?) If the
tableau is open include a picture of one of its open branches in the report
[LoTREC menu: Premodels/Export Premodel...].
4
\item 7. Based on the result of the computation provide the answers to the fol-
lowing problems: 1) the original decision problem; 2) the corresponding
concept satisability problem that you solved with the tableau algorithm.

\end{itemize}

\section{Strategies}
\label{sec:strategies}

\subsection{Blocking}

\subsection{Ordering}

\section{Extensions}
\label{sec:extensions}

2/3

\subsection{Role inverses}

\subsection{Transitive rules}

\subsection{Role hierarchies}

\bibliographystyle{plainnat}
\bibliography{ref}

\end{document}
